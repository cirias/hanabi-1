\section{Background}
\label{background}

Sagar

\subsection{Hanabi}
Actually hanabi here

Hanabi \cite{hanabiboardgame, hanabiwiki} is a cooperative card game in which two or more players act on partial information towards a common goal. Each player can see the cards of all other players, but not their own. On each turn, a player can play a card (without looking at it), discard a card, or give information to another player. The goal is to play as many cards as possible in a specified order.

Figure \ref{fig:hanabisizes} shows 3 sizes of hanabi that we use in our experiments.

\begin{figure}
    \centering
    \begin{tabular}{|c | c | c | c | c | c | c |} \hline
        Name        & Colors & 1s, 2s ... 5s  & Info Tokens & Fuses & Hand Size \\ \hline
        Mini        & 3      & [2, 2, 1, 0, 0]    & 6           & 3     & 3 \\ \hline
        Medium      & 4      & [3, 2, 2, 1, 0] & 8           & 4     & 4 \\ \hline
        Regular     & 5      & [3, 2, 2, 2, 1] & 8 & 4 & 5 \\ \hline
    \end{tabular}
    \caption{Hanabi Game Sizes}
    \label{fig:hanabisizes}
\end{figure}

\subsection{Reinforcement Learning}
% Make sure to describe what a MDP is, what an observation space is, what an
% action space is, and what a reward function is.
