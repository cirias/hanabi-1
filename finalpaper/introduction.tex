\section{Introduction}
\label{intro}

Stephanie

Basic hanabi description here, then talk about general applicability of cooperative AI

Using machine learning to play adversarial games is a well-studied topic. We believe that using machine learning to play a cooperative game will bring about new questions. In Hanabi, for example, the emphasis on cooperation means that it’s advantageous for all players to agree on certain protocols beforehand. One question that we’d like to answer in this project is whether, given a human player that is already following a certain protocol, an AI can be automatically taught to follow the same protocol.

Another question is whether it’s possible to simultaneously train multiple models. The most obvious way to train a model may be to simulate games played with an initial, hardcoded policy, which has deterministic behavior and can therefore act as part of the environment. We believe this will work well for a two-player game, but is unlikely to scale to more players, when more coordination is needed. As part of this project, we would also like to explore the effect of simultaneously training two or more models, possibly with one hard-coded player.
