\section{Introduction}
\label{sec:intro}

In this paper, we present a case study on the feasibility of using machine
learning to perform a cooperative task in which the combined agents have
complete information, but each individual agent has only partial information.
We train an AI to learn to play Hanabi~\cite{hanabiboardgame}, a cooperative
card game in which each player can see everyone else's cards, but not his own.

The cooperative aspect of Hanabi differentiates it from traditional card games,
in which adversaries have inverse objectives. In Hanabi, the basic moves
include playing a card, which may return some points for the group, or giving
information to another agent, which may enable them to play a card on a later
move. This means that points can often only be gained through a cooperative
strategy. In certain cases, for example, it may be more advantageous for an
agent to perform an action that will allow another agent to make progress,
rather than the agent attempting to make progress on its own. We discuss Hanabi
gameplay and strategy in more detail in \secref{background:hanabi}.

A natural fit for learning to play Hanabi is to frame Hanabi gameplay as a
Markov decision process (MDP) [CITE]. MDPs are useful for solving problems that
can be encoded as a state machine, in which each state transition is partly
stochastic and partly influenced by the agent's \emph{action}. Once this state
machine has been encoded, various algorithms~\cite{TRPO} can be used to solve
the MDP to produce a \emph{policy} that decides actions given certain state to
return a maximal \emph{reward}. We discuss MDPs and the methods for solving
such problems in \secref{background:mdp}.

The primary challenge is then to encode Hanabi as an MDP. In
\secref{hanabimdp}, we explore various methods of representing state and reward
to facilitate learning. Representing state properly is challenging because each
agent has only partial information about the entire game state, so some actions
may be directly dependent on some unobserved state. We also adjust the reward
function to match the Hanabi point system and to train an AI that can learn the
rules of gameplay.

Using these Hanabi encodings, we then apply reinforcement learning, a family of
algorithms that can be used to solve MDPs with unknown probabilities, to train
a policy that can play Hanabi. In \secref{learninghanabi}, we discuss the
methods used to train a Hanabi AI, the results of which we present in
\secref{eval}.

In summary, we present a methodology to encode and learn cooperative tasks in
which each individual agent has partial information, with Hanabi as a case
study. We produce a policy that can achieve an average score of \todo{fill out
numbers}.

% Give a basic Hanabi description here. Then talk about general applicability
% of cooperative AI.

% Proposal Intro
% ==============
%
% Using machine learning to play adversarial games is a well-studied topic. We
% believe that using machine learning to play a cooperative game will bring
% about new questions. In Hanabi, for example, the emphasis on cooperation
% means that it’s advantageous for all players to agree on certain protocols
% beforehand. One question that we’d like to answer in this project is whether,
% given a human player that is already following a certain protocol, an AI can
% be automatically taught to follow the same protocol.

% Another question is whether it’s possible to simultaneously train multiple
% models. The most obvious way to train a model may be to simulate games played
% with an initial, hardcoded policy, which has deterministic behavior and can
% therefore act as part of the environment. We believe this will work well for
% a two-player game, but is unlikely to scale to more players, when more
% coordination is needed. As part of this project, we would also like to
% explore the effect of simultaneously training two or more models, possibly
% with one hard-coded player.
