\section{Hanabi as a Markov Decision Procedure [Michael+Stephanie]}\label{sec:hanabimdp}

In this section, we discuss multiple ways to model Hanabi as an Markov decision
process. We also discuss how we implement these various models as OpenAI gym
environments.

\subsection{Space Representations}
% Discuss our two space representations, their strengths, and their weaknesses.

\subsection{Reward Functions}
% Discuss our various reward functions.

\subsection{Implementation}
OpenAI gym is a Python reinforcement learning library which implements a
standard set of reinforcement learning challenges against which researchers can
evaluate their reinforcement learning algorithms~\cite{brockman2016openai}. For
example, OpenAI gym provides implementations of classic reinforcement learning
problems, board games, and Atari games. These implementations follow a common
\emph{environment interface} which specifies the observation space, action
space, transition function, and reward function of an MDP. We have implemented
Hanabi as an OpenAI gym environment which supports all of state representations
and reward functions described above.

% TODO: Maybe we should link to our code here?
% TODO: Maybe shorten this subsection a bit?
