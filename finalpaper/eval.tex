\section{Evaluation}\label{sec:eval}

We trained multiple reinforcement learning algorithms on multiple variants of
Hanabi using dynamic self-learning, guided-learning, and static self-learning.
In this section, we answer questions about how state space representation
affects policy performance, about how well a learned policy can play Hanabi,
about how self-learning differs from guided-learning.

\subsection{Hanabi Variants}\label{sec:eval:hanabivariants}
We trained policies for three variants of Hanabi. The first, \emph{Hanabi}, is
a standard game of Hanabi. The second, \emph{medium Hanabi}, is Hanabi with
only four colors, four numbers (three 1's, two 2's, two 3's, one 4), and four
cards in a hand. The third, \emph{mini Hanabi}, is Hanabi with three colors,
three numbers (two 1's, two 2's, one 3), three cards in a hand, and six
information tokens, and two fuses.

\subsection{Hyperparameter Tuning and Algorithm Selection}
We trained policies with three reinforcement learning algorithms: TRPO, CEM,
and CMA-ES. In order to compare the three algorithms fairly, we first tuned the
hyperparameters of each algorithm. For example, \figref{trpo-tuning} shows the
average reward of a TRPO dynamically self-learned mini Hanabi policy against
the number of training iterations for various values of hyperparameters.

\figref{trpo-tuning} shows that neural network architecture and discount factor
have little effect on performance, while batch size, step size, and state space
do.

\begin{figure}[ht]
  \newcommand{\hyperparamsubfig}[3]{%
    \begin{subfigure}[t]{0.32\textwidth}
      \centering
      \includegraphics[width=\textwidth]{data/sweep/reward_vs_iteration_#1.pdf}
      \caption{#2}\label{fig:#3}
    \end{subfigure}
  }

  \centering

  \hyperparamsubfig{hidden_sizes}{the number and size of hidden layers}{}
  \hyperparamsubfig{batch_size}{batch size}{}
  \hyperparamsubfig{discount}{discount factor}{}

  \hyperparamsubfig{step_size}{step size}{}
  \hyperparamsubfig{reward}{reward function}{}
  \hyperparamsubfig{space}{state space}{}

  \caption{
    Hyperparameter tuning for TRPO trained mini Hanabi policies. The default
    values for each hyperparameter (if not being varied) are as follows: hidden
    sizes=$(16, 16)$, discount=$1$, step size=$0.01$, reward function=constant,
    space=nested.
  }\label{fig:trpo-tuning}
\end{figure}

\begin{figure}[ht]
  \newcommand{\algosubfig}[3]{%
    \begin{subfigure}[b]{0.32\textwidth}
      \centering
      \includegraphics[width=\textwidth]{data/sweep2/#1.pdf}
      \caption{#2}\label{fig:#3}
    \end{subfigure}
  }

  \centering
  \algosubfig{CMA-ES}{CMA-ES}{cmaes}
  \algosubfig{CEM}{CEM}{cem}
  \algosubfig{TRPO}{TRPO}{trpo}
  \caption{Mini Hanabi scores on 1000 mini Hanabi games.}\label{fig:algos}
\end{figure}

We then trained a dynamically self-learned mini Hanabi policy using each
tuned algorithm for 1000 training iterations. Histograms of the policies'
scores on 1000 games of mini Hanabi are shown in \figref{algos}.

The CMA-ES policy performed very poorly with an average score of 1.757. It
performed only marginally better than a completely random policy which obtained
an average score of 1.56.
%
The CEM policy performed slightly better (but still poorly) with an average
score of 2.215.
%
The TRPO policy performed significantly better than either of the other two
policies. Most games resulted in a score of 8, and a significant number of
games resulted in a perfect score.
%
Based on these results, we conducted the rest of our experiments with TRPO.

\subsection{Self Learning vs Guided Learning}
We trained a TRPO policy for 2000 iterations using dynamic self-learning,
guided learning, and static self-learning for mini Hanabi, medium Hanabi, and
Hanabi. We then played each policy against itself for 1000 games. A histogram
of the policies' scores is given in \figref{best}. For comparison, we also show
the scores for our heuristic when played against itself. Note that our
heuristic performed comparably or better than existing heuristic-based
policies~\cite{osawa2015solving,walton2017evaluating}

For mini Hanabi, all three learned policies outperformed the hand-written
heuristic. Moreover, both self-learned policies outperformed the guided-learned
policy. This confirms our hypothesis that performance of a guided-learned
policy is bounded by the performance of the heuristic against which it is
trained. Also note that the 2000-iteration dynamic self-learned policy performs
worse than the 1000-iteration policy presented in \figref{algos}; this
illustrates the inherent randomness in TRPO.

For medium Hanabi and Hanabi, all three learned policies performed worse than
the heuristic. We conjecture that the larger state space of medium Hanabi and
Hanabi require more training time to achieve comparable performance to the
policies trained on mini Hanabi which has a considerably smaller state space.
Moreover, both self-learned policies performed worse than the guided-learned
policy which confirms our hypothesis that guided-learned policies can more
quickly improve while self-learned policies initially struggle to improve. We
are confident that given enough training time, the learned policies would
outperform the heuristic-based policy.

\begin{figure}[ht]
  \newcommand{\bestsubfig}[1]{%
    \begin{subfigure}[b]{0.25\textwidth}
      \centering
      \includegraphics[width=\textwidth]{data/sweep_best/#1.pdf}
    \end{subfigure}
  }

  \centering

  \makebox[\textwidth][c]{%
    \begin{tabular}{|c|c|c|c|c|}
      \cline{2-5}
      \multicolumn{1}{c}{} &
      \multicolumn{1}{|c|}{\textbf{Dynamic Self-Learning}} &
      \textbf{Guided Learning} &
      \textbf{Static Self-Learning} &
      \textbf{Heuristic} \\\hline
      %
      \rotatebox[origin=l]{90}{\textbf{Mini Hanabi}} &
      \bestsubfig{mini_Hanabi_dynamic_self-learned} &
      \bestsubfig{mini_Hanabi_guided-learned} &
      \bestsubfig{mini_Hanabi_static_self-learned} &
      \bestsubfig{mini_Hanabi_heuristic} \\\hline
      %
      \rotatebox[origin=l]{90}{\textbf{Medium Hanabi}} &
      \bestsubfig{medium_Hanabi_dynamic_self-learned} &
      \bestsubfig{medium_Hanabi_guided-learned} &
      \bestsubfig{medium_Hanabi_static_self-learned} &
      \bestsubfig{medium_Hanabi_heuristic} \\\hline
      %
      \rotatebox[origin=l]{90}{\textbf{Hanabi}} &
      \bestsubfig{Hanabi_dynamic_self-learned} &
      \bestsubfig{Hanabi_guided-learned} &
      \bestsubfig{Hanabi_static_self-learned} &
      \bestsubfig{Hanabi_heuristic} \\\hline
    \end{tabular}
  }

  \caption{Hanabi policies.}\label{fig:best}
\end{figure}
