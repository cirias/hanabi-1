\section{Learning to Play Hanabi}\label{sec:learninghanabi}

\todo{Describe the three main ways to learn Hanabi: against itself, against a
heuristic, and against a fixed version of itself.}

\subsection{Heuristics}
\todo{Write.}
% Describe the two heuristics we used to play the game: simple and more
% complex.

The SimpleHeuristic player attempts to mimic the strategy that our best
TRPO-trained-on-itself player learns for mini-hanabi. Each turn, it tries the
following, prioritizing lower numbered cards first:

\begin{itemize}
\item Give info about cards of the given number if there are any that the other player does not know about
\item If a card of the given number is already played for each color, discard any card of that number that we know about in our hand
\item Play a card of the given number.
\end{itemize}

If none of these conditions are satisfied, for example if the other player
knows the number for each of their cards and the heuristic player knows
nothing about their own cards, the player tries to play the zeroeth card in
its hand. We avoid discarding as much as possible, since every discard after
the two costs one point in two player mini-hanabi.




\subsection{Algorithms}
\todo{Write.}
% Describe the algorithms we use (with citations). Mention rllab.
%
%
\subsubsection{Trust Region Policy Optimization (TRPO)}
\cite{TRPO}
\subsubsection{CEM}
\subsubsection{Covariance Matrix Adaptation Evolution Stretegy (CMA-ES)}
~\cite{cmaes}



%\subsubsection{VPG (if space)}


