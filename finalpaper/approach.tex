\section{Approach}
\label{approach}

Here, we describe the various components we implemented to play and learn
Hanabi. Section \ref{eval} compares the performance of the different
state representations, reward functions, algorithms, and training
methods.

\subsection{Overview}
Michael

Talk about how the following pieces fit together + OpenAI


Stephanie 
\subsubsection{Space reps and Reward Functions}

%One limitation of our game engine is that our information tracking
%implementation does not keep information about which colors/numbers a card
%does not map to. For example, in the situation where ``I was told two cards
%are 3s, so the other cards are not 3s,'' our game engine does not remember that
%the other cards are not 3s.


1) Nested

2) Flattened

Talk about difficulty in fixed state rep/action mapping

\subsection{Training}

Make algs/heuristics subsections here, talk about different combos

\subsubsection{Algorithms}

\subsubsection{TRPO}

\cite{TRPO}

\subsubsection{VPG (if space)}
\subsubsection{CEM (if space)}
\subsubsection{CMA-ES (if space)}

\subsection{Heuristics}

We implemented hardcoded heuristic-based Hanabi players to see if we could
guide the previously discussed algorithms into learning a particular strategy.
Below we discuss two of these heuristic-based players.

\subsubsection{Heuristic}

Stephanie

\subsubsection{SimpleHeuristic}

The SimpleHeuristic player attempts to mimic the strategy that our best
TRPO-trained-on-itself player learns for mini-hanabi. Each turn, it tries the
following, prioritizing lower numbered cards first:

\begin{itemize}
\item Give info about cards of the given number if there are any that the other player does not know about
\item If a card of the given number is already played for each color, discard any card of that number that we know about in our hand
\item Play a card of the given number.
\end{itemize}

If none of these conditions are satisfied, for example if the other player
knows the number for each of their cards and the heuristic player knows
nothing about their own cards, the player tries to play the zeroeth card in
its hand. We avoid discarding as much as possible, since every discard after
the two costs one point in two player mini-hanabi.

